\documentclass{article}

\usepackage{tabularx}
\usepackage{booktabs}

\title{CAS 741: Problem Statement\\System of linear solvers}

\author{Devi Prasad Reddy 400153636}

\date{}

\input{../Comments}

\begin{document}

\maketitle

\begin{table}[hp]
\caption{Revision History} \label{TblRevisionHistory}
\begin{tabularx}{\textwidth}{llX}
\toprule
\textbf{Date} & \textbf{Developer(s)} & \textbf{Change}\\
\midrule
SEP15, 2017 & Devi Prasad Reddy Guttapati & Problem Statement\\

\bottomrule
\end{tabularx}
\end{table}


Many of the relationships in nature are linear, that means their effects are
proportional to their causes. For example in Mechanics if we take Newtons second
law of motion, \textbf{F = ma} says that force is proportional to acceleration
and mass is the proportionality constant. For example in Electricity if we take
Ohm's Law, \textbf{V = iR} voltage across the conductor is proportional to the
current flowing though it and resistance is the proportionality constant. These
examples shows us the importance of linear equations and solving them. In matrix
notation the general form of linear equation is

 \centerline{\textbf{Ax = b}}

The most important problem in technical computing is the solution of system
linear equations. So the aim is to create a library that contains a family of  linear solvers
which make the work easy. The software will be used by the undergraduate and
graduate students, and will be very handy for academic tasks. We will solve the
linear equations by giving the input in matrix-vector form. It is developed on
windows platform. Expectations are the Software will be reliable, faster and
accurate.



\end{document}
%\documentclass[handout]{beamer} 
\documentclass[t,12pt,numbers,fleqn]{beamer}
%\documentclass[ignorenonframetext]{beamer}

\newif\ifquestions
%\questionstrue
\questionsfalse

\usepackage{pgfpages} 
\usepackage{hyperref}
\hypersetup{colorlinks=true,
    linkcolor=blue,
    citecolor=blue,
    filecolor=blue,
    urlcolor=blue,
    unicode=false}
\urlstyle{same}

\usepackage{booktabs}
\usepackage{hhline}
\usepackage{multirow}
\usepackage{multicol}
\usepackage{array}

\bibliographystyle{plain}

%\usetheme{Iimenau}

\useoutertheme{split} %so the footline can be seen, without needing pgfpages

%\pgfpagesuselayout{resize to}[letterpaper,border shrink=5mm,landscape]  %if this is uncommented, the hyperref links do not work

\mode<presentation>{}

\input{../def-beamer}

\newcommand{\topic}{04 Requirements Continued DRAFT}

\input{../titlepage}

\begin{document}

\input{../footline}

%%%%%%%%%%%%%%%%%%%%%%%%%%%%%%%%%%%%%%%%%%%%%%%%%%%%%%

\begin{frame}
\frametitle{Requirements}

\bi
\item Administrative details
\item Questions?
\item License and copyright
\item Requirements documentation for scientific computing
\item A new requirements template
\item Advantages of new template and examples
\item The new template from a software engineering perspective
\item Concluding remarks
\item References
\ei
\end{frame}

%%%%%%%%%%%%%%%%%%%%%%%%%%%%%%%%%%%%%%%%%%%%%%%%%%%%%%

\begin{frame}
\frametitle{Administrative Details}

\bi
\item Add me to your GitHub repos, my GitHub id is \texttt{smiths}
\item Linked-In
\item Need schedule for informal presentations and GitHub issues
\ei

\end{frame}

%%%%%%%%%%%%%%%%%%%%%%%%%%%%%%%%%%%%%%%%%%%%%%%%%%%%%%

\begin{frame}
\frametitle{Administrative Details: Deadlines}
~\newline
\begin{tabular}{l l l}
\textbf{Problem Statement} & Week 02 & Sept 15\\
SRS Present & Week 04 & Week of Sept 25\\
SRS & Week 05 & Oct 4\\
V\&V Present & Week 06 & Week of Oct 16\\
V\&V Plan & Week 07 & Oct 25\\
MG Present & Week 08 & Week of Oct 30\\
MG & Week 09 & Nov 8\\
MIS Present & Week 10 & Week of Nov 13\\
MIS & Week 11 & Nov 22\\
Impl.\ Present & Week 12 & Week of Nov 27\\
Final Documentation & Week 13 & Dec 6\\
\end {tabular}

\end{frame}

%%%%%%%%%%%%%%%%%%%%%%%%%%%%%%%%%%%%%%%%%%%%%%%%%%%%%%

\begin{frame}
\frametitle{Questions?}
\begin{itemize}
\item Questions about project choices?
\item Questions about software tools?
\bi
\item git?
\item LaTex?
\ei
\item Partial tex files in the blank project template
\item Problem statement
\end{itemize}
\end{frame}

%%%%%%%%%%%%%%%%%%%%%%%%%%%%%%%%%%%%%%%%%%%%%%%%%%

\begin{frame}

\frametitle{Problems with Developing Quality Scientific Computing Software}

\begin{itemize}

\item Need to know requirements to judge reliability
\item In many cases the only documentation is the code
\item Reuse is not as common as it could be
\begin{itemize}
\item \href{http://www.andrew.cmu.edu/user/sowen/softsurv.html}{\alert{Meshing software survey}}
\item \href{http://www.engr.usask.ca/~macphed/finite/fe_resources/node137.html}{\alert{Public domain finite element
programs}}
\item etc.
\end{itemize}
\item Many people develop ``from scratch''
\item Cannot easily reproduce the work of others
\item Neglect of simple software development technology~\cite{Wilson2006} 
% such as version control software

\end{itemize}

\end{frame}

%%%%%%%%%%%%%%%%%%%%%%%%%%%%%%%%%%%%%%%%%%%%%%%%%%%%%%

\begin{frame}

\frametitle{Adapt Software Engineering Methods}

\begin{itemize}

\item Software engineering improves and quantifies quality %purpose of software engineering
\item Successfully applied in other domains
\begin{itemize}
\item Business and information systems
\item Embedded real time systems
\end{itemize}
\item Systematic engineering process
\item Design through documentation
\item Use of mathematics
\item Reuse of components
\item Warranty rather than a disclaimer %goal of software engineering

\end{itemize}

\end{frame}

%%%%%%%%%%%%%%%%%%%%%%%%%%%%%%%%%%%%%%%%%%%%%%%%%%%%%%

\begin{frame}

\frametitle{Developing Scientific Computing Software}

\begin{itemize}

\item Facilitators
\begin{itemize}
\item One user viewpoint for specifying a physical model
\item Assumptions can be used to distinguish models
\item High potential for reuse
\item Libraries
\item Already mathematical
\end{itemize}

\item Challenges
\begin{itemize}
\item Verification and Validation
\item Acceptance of software engineering methodologies
\item No existing templates or examples %explain that templates are a tool for doc req.
\end{itemize}

\end{itemize}

\end{frame}

%%%%%%%%%%%%%%%%%%%%%%%%%%%%%%%%%%%%%%%%%%%%%%%%%%%%%%

\begin{frame}
\frametitle{Outline of Discussion of Requirements}

\begin{itemize}

\item Background on requirements elicitation, analysis and documentation
\item Why requirements analysis for engineering computation?

\item System Requirements Specification and template for beam analysis software
\begin{itemize}
\item Provides guidelines
\item Eases transition from general to specific
\item Catalyses early consideration of design
\item Reduces ambiguity
\item Identifies range of model applicability
\item Clear documentation of assumptions
\end{itemize}

\end{itemize}
\end{frame}

%%%%%%%%%%%%%%%%%%%%%%%%%%%%%%%%%%%%%%%%%%%%%%%%%%%%%%

\begin{frame}
\frametitle{A Rational Design Process}
%\begin{figure}
\begin{center}
 \includegraphics[width=1.0\textwidth]{../Figures/reqSE.pdf}
\end{center}
%\end{figure}
\end{frame}

%%%%%%%%%%%%%%%%%%%%%%%%%%%%%%%%%%%%%%%%%%%%%%%%%%%%%%

\begin{frame}
\frametitle{Sometimes Include Commonality Analysis}
%\begin{figure}
\begin{center}
 \includegraphics[width=1.0\textwidth]{../Figures/Waterfall.pdf}
\end{center}
%\end{figure}
\end{frame}

%%%%%%%%%%%%%%%%%%%%%%%%%%%%%%%%%%%%%%%%%%%%%%%%%%%%%%

\begin{frame}
\frametitle{Software Requirements Activities}
\begin{itemize}
\item A software requirement is a description of how the system should behave, or of a system property or attribute
\item Requirements should be unambiguous, complete, consistent, modifiable, verifiable and traceable
\item Requirements should express ``What'' not ``How''
\item Formal versus informal specification
\item Functional versus nonfunctional requirements
\item Software requirements specification (SRS)
\item Requirements template
\end{itemize}
\end{frame}

%%%%%%%%%%%%%%%%%%%%%%%%%%%%%%%%%%%%%%%%%%%%%%%%%%%%%%

\begin{frame}
\frametitle{Why Requirements Analysis?}
%\begin{figure}
\begin{center}
 \includegraphics[width=1.0\textwidth]{../Figures/StagesInSciCompErrors.pdf}
\end{center}
%\end{figure}
\end{frame}

%%%%%%%%%%%%%%%%%%%%%%%%%%%%%%%%%%%%%%%%%%%%%%%%%%%%%%

\begin{frame}
\frametitle{Beam Analysis Software}
~\newline
~\newline
\begin{center}
 \includegraphics[width=1.0\textwidth]{../Figures/beamFBD.pdf}
\end{center}
\end{frame}

%%%%%%%%%%%%%%%%%%%%%%%%%%%%%%%%%%%%%%%%%%%%%%%%%%%%%%

\begin{frame}
\frametitle{Proposed Template}

\scalebox{0.85}{
\begin{minipage}{1.2\textwidth}
\begin{enumerate}

\item Reference Material: a) Table of Symbols ...

\item Introduction: a) Purpose of the Document; b) Scope of the Software Product; c) Organization of the Document.

\item General System Description: a) System Context; b) User Characteristics; c) System Constraints.

\item Specific System Description:
\begin{enumerate}
\item Problem Description: i) Background Overview ...
\item Solution specification: i) Assumptions; ii) Theoretical Models; ...
\item Non-functional Requirements: i) Accuracy of Input Data; ii) Sensitivity ...
\end{enumerate}

\item{Traceability Matrix}

\item List of Possible Changes in the Requirements

\item{Values of Auxiliary Constants}

\end{enumerate}
\end{minipage}
}
\end{frame}

%%%%%%%%%%%%%%%%%%%%%%%%%%%%%%%%%%%%%%%%%%%%%%%%%%%%%%

\begin{frame}
\frametitle{Provides Guidance}
\begin{itemize}
\item Details will not be overlooked, facilitates multidisciplinary collaboration
\item Encourages a systematic process
\item Acts as a checklist
\item Separation of concerns
\begin{itemize}
\item Discuss purpose separately from organization
\item Functional requirements separate from non-functional
%\begin{itemize}
%\item solve for forces
%\item system responds within 1 second
%\end{itemize}
\end{itemize}
\item Labels for cross-referencing
\begin{itemize}
\item Sections, physical system description, goal statements, assumptions, etc.
\item PS1.a ``the shape of the beam is long and thin''
\end{itemize}
%\item Use of parameters instead of explicit values
\end{itemize}
\end{frame}

%%%%%%%%%%%%%%%%%%%%%%%%%%%%%%%%%%%%%%%%%%%%%%%%%%%%%%

\begin{frame}
\frametitle{Eases Transition from General to Specific}
\begin{itemize}
\item ``Big picture'' first followed by details
\item Facilitates reuse
\item ``Introduction'' to ``General System Description'' to ``Specific System Description''
\item Refinement of abstract goals to theoretical model to instanced model
\begin{itemize}
\item \textbf{G1}. Solve for the unknown external forces applied to the beam
\item $ \textbf{T1}~ 
\textrm{$\sum{F_{xi}} = 0$,}~  
\textrm{$\sum{F_{yi}} = 0$,}~
\textrm{$\sum{M_i} = 0$}$
\item \textbf{M1} \textrm{$F_{ax} - F_1\cdot \cos\theta_3 - F_2\cdot \cos\theta_4 - F_{bx} = 0$}
\end{itemize}
\end{itemize}
\end{frame}

%%%%%%%%%%%%%%%%%%%%%%%%%%%%%%%%%%%%%%%%%%%%%%%%%%%%%%

\begin{frame}
\frametitle{Ensures Special Cases are Considered}
\scalebox{0.6}{
\begin{tabular}{| p{3.8cm} | p{1.7cm} | p{0.05cm} | p{9.0cm} | p{1.8cm} |}
\multicolumn{3}{c}{} & \multicolumn{2}{>{\large}c}{$H_1$} \\
\hhline{|~|~|~|-|-|}
\multicolumn{3}{c}{} & \multicolumn{1}{|c|}{$S_{GET} = S_{sym} - S_{unkF}$} & $S_{GET} \ne  (S_{sym} - S_{unkF})$ \\
\hhline{|~|~|~|-|-|}

\hhline{|-|-|~|-|-|} $S_{unkF} \notin \mathbb{P}_3$ & - &  & $(ErrorMsg'=InvalidUnknown)$ \newline
$\land ChangeOnly(ErrorMsg)$ &
\multirow{9}{2cm}{$FALSE$} \\
\hhline{|-|-|~|-|~|} $S_{unkF} = \newline \{@{F_{ax}}, @{F_{bx}}, @{F_{ay}} \}$ & - & & $ErrorMsg'=NoSolution$ \newline
$\land ChangeOnly(ErrorMsg)$ & \\
%\hhline{|-|-|~|-|~|} $S_{unkF} = \newline \{@{F_{ax}}, @{F_{bx}}, @{F_{by}} \}$ & - & & $ErrorMsg'=NoSolution$ \newline
%$\land ChangeOnly(ErrorMsg)$ & \\
%\hhline{|-|-|~|-|~|} $S_{unkF} = \newline \{@{F_{ax}}, @{F_{bx}}, @{F_1} \}$ & - & & $ErrorMsg'=NoSolution$ \newline
%$\land ChangeOnly(ErrorMsg)$ & \\
%\hhline{|-|-|~|-|~|} $S_{unkF} = \newline \{@{F_{ax}}, @{F_{bx}}, @{F_2} \}$ & - & & $ErrorMsg'=NoSolution$ \newline
%$\land ChangeOnly(ErrorMsg)$ & \\
\hhline{|-|-|~|-|~|}
%\multirow{3}{4.2cm}
{$S_{unkF} = \newline \{@{F_{ax}}, @{F_{ay}}, @{F_1}\}$} & 
$x_1 \ne 0 $ \newline
$\land~\theta_3 \ne 0$ \newline
$\land~\theta_3 \ne 180$
& & 
$F_{ax}' = $\newline
$\frac{-\cos\theta_3 F_2 x_2 \sin\theta_4 + \cos\theta_3 F_{by} L + F_2 \cos\theta_4 x_1 \sin\theta_3
+ F_{bx} x_1 \sin\theta_3}{x_1 \sin\theta_3}$\newline
$\land$\newline
$F_{ay}' = -\frac{F_2 x_2 \sin\theta_4 - F_{by} L - F_2 \sin\theta_4 x_1 + F_{by} x_1}{x_1}$\newline
{$\land~F_1' = \frac{-F_2 x_2 \sin\theta_4 + F_{by} L}{x_1 \sin\theta_3} \land ChangeOnly(S_{unkF})$}
& \\
\hhline{|~|-|~|-|~|} & $otherwise$ & & $(ErrorMsg'=Indeterminant)$\newline
$\land ChangeOnly(ErrorMsg)$ & \\
\hhline{|-|-|~|-|-|}

\multicolumn{5}{c}{} \\
\multicolumn{3}{>{\large}c}{$H_2$} & \multicolumn{2}{>{\large}c}{$G$} \\
\end{tabular} }
\end{frame}

%%%%%%%%%%%%%%%%%%%%%%%%%%%%%%%%%%%%%%%%%%%%%%%%%%%%%%

\begin{frame}
\frametitle{Catalyses Early Consideration of Design}
\begin{itemize}
\item Identification of significant issues early will improve the design
\item Section for considering sensitivity
\begin{itemize}
\item Conditioning?
\item Buckling of beam
\end{itemize}
\item Non-functional requirements
\begin{itemize}
\item Tradeoffs in design
\item Speed efficiency versus accuracy
\end{itemize}
\item Tolerance allowed for solution: $|\sum{F_{xi}}| / \sqrt{\sum{F_{xi}}^2} \le \epsilon$
\item Solution validation strategies
\item List of possible changes in requirements
\end{itemize}
\end{frame}

%%%%%%%%%%%%%%%%%%%%%%%%%%%%%%%%%%%%%%%%%%%%%%%%%%%%%%

\begin{frame}
\frametitle{Reduces Ambiguity}
\begin{itemize}
\item Unambiguous requirements allow communication between experts, requirements review, designers do not have to
make arbitrary decisions
\item Tabular expressions allow automatic verification of completeness
\item Table of symbols
\item Abbreviations and acronyms
\item Scope of software product and system context
\item User characteristics
\item Terminology definition and data definition
\item Ends arguments about the relative merits of different designs
\end{itemize}
\end{frame}

%%%%%%%%%%%%%%%%%%%%%%%%%%%%%%%%%%%%%%%%%%%%%%%%%%%%%%

\begin{frame}
\frametitle{Identifies Range of Model Applicability}
\begin{itemize}
\item Clear documentation as to when model applies
\item Can make the design specific to the problem
\item Input data constraints are identified
\begin{itemize}
\item Physically meaningful: $0 \leq x_1 \leq L$
\item Maintain physical description: PS1.a, $0 < h \leq 0.1 L$
\item Reasonable requirements: $0 \leq \theta_3 \leq 180$
\end{itemize}
\item The constraints for each variable are documented by tables, which are later composed together
\item $(min_f \le |F_{ax}| \le max_f) 
\land (|F_{ax}| \ne 0) \Rightarrow \forall ({FF}|{@{FF} \in S_F} \cdot {FF \ne 0
\land  \frac{max\{{|F_{ax}|,|FF|}\}}{min\{{|F_{ax}|, |FF|}\}} \le 10 ^ {r_f}})$
\end{itemize}
\end{frame}

%%%%%%%%%%%%%%%%%%%%%%%%%%%%%%%%%%%%%%%%%%%%%%%%%%%%%%
\begin{frame}

\frametitle{Summary of Variables}

\begin{table}
\begin{center}
\scalebox{0.9}{
\begin{tabular}{|l|l|p{3.0cm}|p{3.3cm}|l|}
\multicolumn{5}{c}{} \\
\hline
\textbf{Var} & \textbf{Type} & \textbf{Physical\newline Constraints} & \textbf{System\newline Constraints} &
\textbf{Prop} \\
\hline $x$ & $Real$ & $x\ge 0 \land x\le L$ & $min_d \le x \le max_d$ & NIV \\
\hline $x_1$ & $Real$ & $x_1\ge0 \land x_1\le L$ & $min_d \le x_1 \le max_d$ & IN \\
\hline $x_2$ & $Real$ & $x_2\ge0 \land x_2\le L$ & $min_d \le x_2 \le max_d$ & IN \\
\hline $e$ & $Real$ & $e>0 \land e \le h$ & $min_e \le e \le max_e$ & IN \\
\hline $h$ & $Real$ & $h>0 \land h\le 0.1L$ & $min_h \le h \le max_h$ & IN \\
\hline $L$ & $Real$ & $L>0$ & $min_d \le L \le max_d$ & IN \\
\hline $E$ & $Real$ & $E>0$ & $min_E \le E \le max_E$ & IN \\
\hline $\theta_3$ & $Real$ & $-\infty < \theta_3 < +\infty$ & $0 \le \theta_3 \le 180$ & IN \\
\hline $\theta_4$ & $Real$ & $-\infty < \theta_4 < +\infty$ & $0 \le \theta_4 \le 180$ & IN \\
\hline $V$ & $Real$ & $-\infty < V < +\infty$ & - & OUT \\
\hline $M$ & $Real$ & $-\infty < M < +\infty$ & - & OUT \\
\hline $y$ & $Real$ & $-\infty < y < +\infty$ & - & OUT \\
\hline $...$ & $...$ & $...$ & ... & ... \\
\hline
\end{tabular} }
\end{center}
\end{table}

\end{frame}

%%%%%%%%%%%%%%%%%%%%%%%%%%%%%%%%%%%%%%%%%%%%%%%%%%%%%%

\begin{frame}
\frametitle{Clear Documentation of Assumptions}
\scalebox{0.82}{
\begin{tabular}{| p{1.3cm} | p{1.3cm} | l | l | l | l | l | l | l | l | l | l | l | l |}
\hhline{--------------}
Phy. Sys. /Goal & Data /Model & \multicolumn{10}{c|}{Assumption} & \multicolumn{2}{c|}{Model} \\
\hhline{~~------------}
&  & A1 & A2 & ... & A4 & ... & A8 & A9 & A10 & ...  & A14 & \textbf{M1} & ... \\
\hhline{--------------}
\textbf{G1} & \textbf{T1} & $\surd$ & & ... &  & ... & $\surd$ & $\surd$ &  & ... & & $\surd$ & ...\\
\hhline{--------------}
\textbf{G2} & \textbf{T2} & $\surd$ & & ... & &... & $\surd$ & $\surd$ &  & ... & &  & ... \\
\hhline{--------------}
\textbf{G3} & \textbf{T3} & $\surd$ & & ... &  &... &  & $\surd$ & $\surd$ & ... & &  & ...\\
\hhline{--------------}
~ & \textbf{M1} &  & $\surd$ & ...  &  & ... &  & &  & ... & & $\surd$ &... \\
\hhline{--------------}
PS1.a & $L$ &  & &... & &...  & & & $\surd$  & ... & & ... & ... \\
\hhline{--------------}
... & ... & ... & ... & ... & ... & ... & ... & ... & ... & ... & ... & ... & ... \\
\hhline{--------------}
\end{tabular}
}
~\newline
~\newline
\textbf{A10}. The deflection of the beam is caused by bending moment only, the shear does not contribute.\\
%\textbf{A15}. The beam behaves as a rigid body
\end{frame}

%%%%%%%%%%%%%%%%%%%%%%%%%%%%%%%%%%%%%%%%%%%%%%%%%%%%%%

\frame{\frametitle{More on the Template}
\begin{itemize}%[<+-| alert@+>]%[iacolor=gray]
\item Why a new template?
\item The new template
\begin{itemize}
\item Overview of changes from existing templates
\item Goal $\rightarrow$ Theoretical Model $\rightarrow$ Instanced Model hierarchy
\item Traceability matrix
\item System behaviour, including input constraints
\end{itemize}
\end{itemize}
}

%%%%%%%%%%%%%%%%%%%%%%%%%%%%%%%%%%%%%%%%%%%%%%%%%%%%%%

\frame{\frametitle{Why a New Template?}
\begin{enumerate}%[<+-| alert@+>]%[iacolor=gray]
%\item Reasons for a new template also form principles for its design
\item One user viewpoint for the physical model
\item Assumptions distinguish models
\item High potential for reuse of functional requirements
\item Characteristic hierarchical nature facilitates change
\item Continuous mathematics presents a challenge
\end{enumerate}
}

%%%%%%%%%%%%%%%%%%%%%%%%%%%%%%%%%%%%%%%%%%%%%%%%%%%%%%

\frame{\frametitle{Overview of the New Template}

\begin{itemize}

\item{Reference Material}

\item{Introduction:} 
{a) Purpose of the Document}
{b) Scope of the Software Product}
{c) Organization of the Document}

\item General System Description:
{a) System Context}
{b) User Characteristics}
{c) System Constraints}

\item \structure<2->{Specific System Description:
a) Problem Description 
b) Solution Characteristics Specification
c) Non-functional Requirements}

\item{Other System Issues}

\item \structure<2->{Traceability Matrix}

\item List of Possible Changes in the Requirements

\item{Values of Auxiliary Constants}

\item{References}

\end{itemize}
}

%%%%%%%%%%%%%%%%%%%%%%%%%%%%%%%%%%%%%%%%%%%%%%%%%%%%%%

\begin{frame}
\frametitle{Excerpts from Specific System Description}

\begin{itemize}

\item Problem Description
\begin{itemize}
\item Physical system description (\textbf{PS}) 
\item Goals (\textbf{G})
\end{itemize}
 
\item Solution Characteristics Specification
\begin{itemize}
\item Assumptions (\textbf{A})
\item Theoretical models (\textbf{T})
\item Data definitions
\item Instanced models (\textbf{M})
\item Data constraints
\item System behaviour
\end{itemize}

\item Non-functional Requirements
\begin{itemize}
\item Accuracy of input data
\item Sensitivity of the model
\item Tolerance of the solution
\item Solution validation strategies
\end{itemize}

\end{itemize}

\end{frame}

%%%%%%%%%%%%%%%%%%%%%%%%%%%%%%%%%%%%%%%%%%%%%%%%%%%%%%

\begin{frame}
\frametitle{Refinement from Abstract to Concrete}

\begin{overlayarea}{\textwidth}{5.3cm}
\begin{figure}[H]
\includegraphics<1>[scale=0.41]{../Figures/RefinementHierarchy.pdf}
\includegraphics<2>[scale=0.41]{../Figures/RefinementG1.pdf}
\includegraphics<3>[scale=0.41]{../Figures/RefinementT11.pdf}
\includegraphics<4>[scale=0.41]{../Figures/RefinementM111.pdf}
\includegraphics<5>[scale=0.41]{../Figures/RefinementT12.pdf}
\end{figure}
\end{overlayarea}

\begin{overlayarea}{\textwidth}{1cm}

\only<2>{\textbf{G1}: Solve for unknown forces}

\only<3>{
\begin{center} 
$%\begin{displaymath}
\mathbf{(T1_1)}~\left\{ 
\begin{array}{lll}
\textrm{$\sum{F_{xi}} = 0$}\\  
\textrm{$\sum{F_{yi}} = 0$}\\
\textrm{$\sum{M_i} = 0$}\\
\end{array} \right. $%\end{displaymath}
\end{center}
}

\only<4>{
\begin{center} $%\begin{displaymath}
\textbf{(M1)}~\left\{ 
\begin{array}{lll}
\textrm{$F_{ax} - F_1\cdot \cos\theta_3 - F_2\cdot \cos\theta_4 - F_{bx} = 0$} \\ 
\textrm{$F_{ay} - F_1\cdot \sin\theta_3 - F_2\cdot \sin\theta_4 + F_{by} = 0$}\\
\textrm{$- F_1\cdot x_1\sin\theta_3 - F_2\cdot x_2\sin\theta_4 + F_{by}\cdot L = 0$}\\
\end{array} \right. 
$%\end{displaymath}
\end{center}
}

\only<5>{
The virtual work done by all the external forces and couples acting on the system is zero for each independent virtual
displacement of the system, or mathematically $\delta U = 0$
}
\end{overlayarea}

\end{frame}

%%%%%%%%%%%%%%%%%%%%%%%%%%%%%%%%%%%%%%%%%%%%%%%%%%%%%%

\begin{frame}
\frametitle{Other goals and models}
\begin{itemize}
\item \textbf{G2}: Solve for the functions of shear force and bending moment along the beam
\item \textbf{G3}: Solve for the function of deflection along the beam
\item $\mathbf{T3_1}$: $\frac{d^2 y}{d x^2} = \frac{M}{EI}$, $y(0) = y(L) = 0$
\item $\mathbf{T3_2}$: $y$ determined by moment area method
\item $\mathbf{T3_3}$: $y$ determined using Castigliano's theorem
\item $\mathbf{M3_{11}}$: $y = \frac{12 \int_0^L (\int_0^L M dx) dx}{Eeh^3}$, $y(0) = y(L) = 0$
\end{itemize}
\end{frame}

%%%%%%%%%%%%%%%%%%%%%%%%%%%%%%%%%%%%%%%%%%%%%%%%%%%%%%

\begin{frame}
\frametitle{Kreyman and Parnas Five Variable Model}
\begin{itemize}
\item An alternative approach
\item Unfortunately the numerical algorithm is not hidden in the requirements specification
\item The analogy with real-time systems leads to some confusion
\end{itemize}
\end{frame}

%%%%%%%%%%%%%%%%%%%%%%%%%%%%%%%%%%%%%%%%%%%%%%%%%%%%%%

\begin{frame}
\frametitle{Examples}
\begin{itemize}
\item \href{https://github.com/smiths/swhs}{Solar Water Heating System}
\item \href{https://github.com/JacquesCarette/literate-scientific-software/tree/master/CaseStudies/glass/Documentation/SRS}{GlassBR}
\end{itemize}
\end{frame}

%%%%%%%%%%%%%%%%%%%%%%%%%%%%%%%%%%%%%%%%%%%%%%%%%%%%%%

\begin{frame}
\frametitle{Summary of Template}
\begin{itemize}
\item Quality is a concern for scientific computing software
\item Software engineering methodologies can help
\item Motivated, justified and illustrated a method of writing requirements specification for engineering computation
to improve reliability
\item Also improve quality with respect to usability, verifiability, maintainability, reusability and portability
\item Tabular expressions to reduce ambiguity, encourage systematic approach
\item Conclusions can be generalized because other computation problems follow the same pattern of \emph{Input} then
\emph{Calculate} then \emph{Output}
\item Benefits of approach should increase as the number of details and the number of people involved increase
\end{itemize}
\end{frame}

%%%%%%%%%%%%%%%%%%%%%%%%%%%%%%%%%%%%%%%%%%%%%%%%%%%%%%

\begin{frame}
\frametitle{Summary of Template (Continued)}
\begin{itemize}
\item A new template for scientific computing has been developed
\item Characteristics of scientific software guided the design
\item Designed for reuse
\item Functional requirements split into ``Problem Description'' and ``Solution Characteristics Specification''
\item Traceability matrix
\item Addresses nonfunctional requirements (but room for improvement)
\end{itemize}

\end{frame}

%%%%%%%%%%%%%%%%%%%%%%%%%%%%%%%%%%%%%%%%%%%%%%%%%%%%%%

\begin{frame}
\frametitle{Specification Qualities}

\begin{itemize}

\item \structure{What are the important qualities for a specification?}

\end{itemize}

\end{frame}

%%%%%%%%%%%%%%%%%%%%%%%%%%%%%%%%%%%%%%%%%%%%%%%%%%%%%%

\begin{frame}
\frametitle{Specification Qualities}

\begin{itemize}
\item The qualities we previously discussed (usability, maintainability,
  reusability, verifiability etc.)
\item Clear, unambiguous,  understandable
\item Consistent
\item Complete
\begin{itemize}
\item Internal completeness
\item External completeness
\end{itemize}
\item Incremental
\item Validatable
\item Abstract
\item Traceable
\end{itemize}

Summarized in \cite[p.\ 406]{SmithAndKoothoor2016}

\end{frame}

%%%%%%%%%%%%%%%%%%%%%%%%%%%%%%%%%%%%%%%%%%%%%%%%%%%%%%

\begin{frame}
\frametitle{Clear, Unambiguous, Understandable}

\begin{itemize}

\item Specification fragment for a word-processor
\begin{itemize}
\item \structure{Selecting is the process of designating 
areas of the document that you want to 
work on. Most editing and formatting 
actions require two steps: first you 
select what you want to work on, 
such as text or graphics; then you 
initiate the appropriate action.}
\end{itemize}
\item What are the potential problems with this specification?
\begin{itemize}
\item<2-> {\alert{Can an area be scattered?}}
\item<2->{\alert{Can both text and graphics be selected?}}
\end{itemize}
\end{itemize}

\end{frame}

%%%%%%%%%%%%%%%%%%%%%%%%%%%%%%%%%%%%%%%%%%%%%%%%%%%%%%

\begin{frame}
\frametitle{Clear, Unambiguous, Understandable}

\begin{itemize}

\item Specification fragment from a real safety-critical system
\begin{itemize}
\item \structure{The message must be triplicated. The three
copies must be forwarded through three 
different physical channels. The receiver 
accepts the message on the basis of a 
two-out-of-three voting policy.}
\end{itemize}
\item What is a potential problems with this specification?
\begin{itemize}
\item<2-> {\alert{Can a message be accepted as soon as we receive 2 out of 3 identical copies, or do we need to wait
for receipt of the 3rd}}
\end{itemize}
\end{itemize}

\end{frame}

%%%%%%%%%%%%%%%%%%%%%%%%%%%%%%%%%%%%%%%%%%%%%%%%%%%%%%

\begin{frame}
\frametitle{Unambiguous, Validatable}

\begin{itemize}

\item Specification fragment for an end-user program
\begin{itemize}
\item \structure{The program shall be user friendly.}
\end{itemize}
\item What is a potential problems with this specification?
\begin{itemize}
\item<2-> {\alert{What does it mean to be user friendly?}}
\item<2-> {\alert{Who is a typical user?}}
\item<2-> {\alert{How would you measure success or failure in meeting this requirement?}}
\end{itemize}

\end{itemize}

\end{frame}

%%%%%%%%%%%%%%%%%%%%%%%%%%%%%%%%%%%%%%%%%%%%%%%%%%%%%%

\begin{frame}
\frametitle{Unambiguous, Validatable}

\begin{itemize}

\item Specification fragment for a linear solver
\begin{itemize}
\item \structure{Given $A$ and $b$, solve the  linear system $A x = b$ for $x$, such that the error in any entry of
$x$ is less than 5 \%.}
\end{itemize}
\item What is a potential problems with this specification?
\begin{itemize}
\item<2-> {\alert{Is $A$ constrained to be square?}}
\item<2-> {\alert{Can $A$ be singular?}}
\item<2-> {\alert{Even if the problem is made completely unambiguous, the requirement cannot be validated.}}
\end{itemize}

\end{itemize}

\end{frame}

%%%%%%%%%%%%%%%%%%%%%%%%%%%%%%%%%%%%%%%%%%%%%%%%%%%%%%

\begin{frame}
\frametitle{Consistent}

\begin{itemize}

\item Specification fragment for a word-processor
\begin{itemize}
\item \structure{The whole text should be kept in lines 
of equal length. The length is specified 
by the user. Unless the user gives an 
explicit hyphenation command, 
a carriage return should occur only 
at the end of a word.}
\end{itemize}
\item What is a potential problems with this specification?
\begin{itemize}
\item<2-> {\alert{What if the length of a word exceeds the length of the line?}}
\end{itemize}

\end{itemize}

\end{frame}

%%%%%%%%%%%%%%%%%%%%%%%%%%%%%%%%%%%%%%%%%%%%%%%%%%%%%%

\begin{frame}
\frametitle{Same Symbol/Term Different Meaning}

\begin{itemize}

\item \structure{Can you think of some symbols/terms that have different
    meanings depending on the context?}

\end{itemize}

\end{frame}

%%%%%%%%%%%%%%%%%%%%%%%%%%%%%%%%%%%%%%%%%%%%%%%%%%%%%%

\begin{frame}
\frametitle{Consistent}

\begin{itemize}

\item Language and terminology must be consistent within the specification
\item Potential problem with homonyms, for instance consider the symbol $\sigma$
\begin{itemize}
\item Represents standard deviation
\item Represents stress
\item Represents the Stefan-Boltzmann constant (for radiative heat transfer)
\end{itemize}
\item Changing the symbol may be necessary for consistency, but it could adversely effect understandability
\item Potential problem with synonyms
\begin{itemize}
\item Externally funded graduate students, versus eligible graduate students, versus non-VISA students
%ask who would think about graduate school?
\item Material behaviour model versus constitutive equation
\end{itemize}
\end{itemize}

\end{frame}

%%%%%%%%%%%%%%%%%%%%%%%%%%%%%%%%%%%%%%%%%%%%%%%%%%%%%%

\begin{frame}
\frametitle{Complete}

\begin{itemize}

\item Internal completeness
\begin{itemize}
\item The specification must define any new concept or terminology that it uses
\begin{itemize}
\item A glossary is helpful for this purpose
\end{itemize}
\end{itemize}
\item External completeness
\begin{itemize}
\item The specification must document all the needed requirements
\begin{itemize}
\item Difficulty: when should one stop?
\end{itemize}
\end{itemize}

\end{itemize}

\end{frame}

%%%%%%%%%%%%%%%%%%%%%%%%%%%%%%%%%%%%%%%%%%%%%%%%%%%%%%

\begin{frame}
\frametitle{Incremental}

\begin{itemize}

\item Referring to the specification process
\begin{itemize}
\item Start from a sketchy document and progressively add details
\item A document template can help with this
\end{itemize}
\item Referring to the specification document
\begin{itemize}
\item Document is structured and can be understood in increments
\item Again a document template can help with this
\end{itemize}

\end{itemize}

\end{frame}

%%%%%%%%%%%%%%%%%%%%%%%%%%%%%%%%%%%%%%%%%%%%%%%%%%%%%%

\begin{frame}
\frametitle{Traceable}

\begin{itemize}

\item Definition
\item Common sense suggests traceability improves maintainability

\end{itemize}

\end{frame}

%%%%%%%%%%%%%%%%%%%%%%%%%%%%%%%%%%%%%%%%%%%%%%%%%%%%%%

\begin{frame}[allowframebreaks]
\frametitle{References}

\nocite{Lai2004, SmithEtAl2004, SmithAndLai2005, IEEE1998, IEEE2000,
  IEEE2000_ThayerAndDorfman, RobertsonAndRobertson1999Vol, DesharnaisEtAl2004,
  JanickiAndKhedri2001, GhezziEtAl2003, KreymanAndParnas2002,
  ParnasAndClements1986, Segal2007_Dagsthul, Segal2008, Segal2007, Segal2005, Dubois2002}

\bibliography{../../ReferenceMaterial/References}

\end{frame}

%%%%%%%%%%%%%%%%%%%%%%%%%%%%%%%%%%%%%%%%%%%%%%%%%%%%%%

\end{document}
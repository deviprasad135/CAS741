\documentclass[12pt, titlepage]{article}

\usepackage{booktabs}
\usepackage{tabularx}
\usepackage{hyperref}
\usepackage{graphicx}
\usepackage{amsmath, mathtools}
\usepackage{amsfonts}
\usepackage{amssymb}
\usepackage{graphicx}
\usepackage{colortbl}
\usepackage{xr}
\usepackage{hyperref}
\usepackage{longtable}
\usepackage{xfrac}
\usepackage{tabularx}
\usepackage{float}
\usepackage{siunitx}
\usepackage{booktabs}
\usepackage{caption}
\usepackage{pdflscape}
\usepackage{afterpage}
\hypersetup{
    colorlinks,
    citecolor=black,
    filecolor=black,
    linkcolor=red,
    urlcolor=blue
}
\usepackage[round]{natbib}

\newcounter{tnum}
\newcommand{\tthetestnum}{\thetnum}
\newcommand{\tref}[1]{T\ref{#1}}

%%% Comments

\usepackage{color}

\newif\ifcomments\commentstrue

\ifcomments
\newcommand{\authornote}[3]{\textcolor{#1}{[#3 ---#2]}}
\newcommand{\todo}[1]{\textcolor{red}{[TODO: #1]}}
\else
\newcommand{\authornote}[3]{}
\newcommand{\todo}[1]{}
\fi

\newcommand{\wss}[1]{\authornote{blue}{SS}{#1}}
\newcommand{\an}[1]{\authornote{magenta}{Author}{#1}}


\begin{document}

\title{Test Plan for the Library of Linear Algebraic Equation Solver }

\author{Devi Prasad Reddy Guttapati}

\date{\today}





\maketitle

\pagenumbering{roman}


]
\section{\bf Revision History}
\begin{tabularx}{\textwidth}{p{3cm}p{2cm}X}
\toprule {\bf Date} & {\bf Version} & {\bf Notes}\\
\midrule
\today & 1.0 & Initial draft\\

\bottomrule
\end{tabularx}


\newpage


\section{Symbols, Abbreviations and Acronyms}

\renewcommand{\arraystretch}{1.2}
\begin{tabular}{l l} 
  \toprule		
  \textbf{symbol} & \textbf{description}\\
  \midrule 
  T & Test\\
O & Output\\
CA & Commonality Analysis\\
IM & Instance Module\\
$\epsilon$ & Difference\\


  \bottomrule
\end{tabular}\\

\newpage

\tableofcontents

\listoftables




\newpage


\pagenumbering{arabic}


\section{General Information}

The following section gives an overview of the Verification and Validation ( V
\& V ) plan for the Library of Linear Algebraic Equation Solver. This section
also explains the purpose, scope and overview of the document. The full documentation can be found at the mentioned link below:

\url{https://github.com/deviprasad135/CAS741}

\subsection{Purpose}

The purpose of this document is to plan the Verification and Validation process
for the Library of Linear Algebraic Equation Solver. The main purpose of this
document is to check whether the Library of Linear Algebraic Equation Solver
meets the specifications and fulfill its intended purpose. This document will be
used as the reference and guidance for testing the Library of Linear Algebraic
Equation Solver.




\subsection{Scope}

The scope of testing is limited to Library of Linear Algebraic Equation Solver.
The library includes two linear algebraic solving functions. The scope is
limited to these two solving functions. The programming language used is R. 

See SRS Documentation at:  \url{https://github.com/deviprasad135/CAS741/blob/master/Doc/SRS/CA.pdf}


\subsection{Overview of Document}

The following sections provides in depth information about the V \& V of Library
of Linear Algebraic Equation Solver. The following sections also provides
information about automated testing approach, testing tools. Test cases for
system testing and unit testing are provided.

\section{Plan}
	
\subsection{Software Description}

Software which is tested is Library of Linear Algebraic Equation Solver. The
library contains two linear algebraic equation solving algorithms. Given the
initial values of algebraic equation $A$ and $B$, the program calculates the final
value $x$ by using numerical methods.

\subsection{Test Team}

Devi Prasad Reddy Guttapati

\subsection{Automated Testing Approach}

Automated unit testing will be done for the Library of Linear Algebraic Equation
Solver. Unit Testing Framework and R's Test Class will be used in a combination
for automated testing. Syntax checking will be done automatically by the
compiler. The aim of testing is 100\%  code coverage .

\subsection{Verification Tools}

The verification tools which will be used are as follows :

\begin{enumerate}

\item The programming language used is R. Comparison is done between the
RStusio's optR library functional programs with the Library of Linear Algebraic Equation
Solver by the Unit Testing Framework designed in RStudio. The following
equation is used for comparison.

\begin{center}
 $\epsilon_\text{rel} = \text{norm} = \frac{||x_\text{R} - x_\text{LAES}||}{||x_\text{R}||}$
\end{center}




\item The framework for automated system testing is provided by using RStudio's Test Class .

\item For program debugging and for checking syntax errors the Rstudio's IDE will be used as a Static Analyzer tool .

\item covr which is an RStudio's library is used for code coverage.

\item RStudio's unit test library will be used for performing unit tests on the code. 
\end{enumerate}



\subsection{Non-Testing Based Verification}
		
Not Applicable

\section{System Test Description}

System Test is done to verify whether the goals mentioned in Commonality
Analysis are achieved. Input and Output analysis is used to test the system as a
whole by black box testing approach.
	
\subsection{Tests for Functional Requirements}

\subsubsection{Calculation Tests}
		
\paragraph{Gaussian Elimination Method }

\begin{enumerate}

\item{\textbf{T-\refstepcounter{tnum}\thetnum \label{t-gauss
elimination_simple}: Simple Linear system involving two equations and two
variables}}


Type: Functional, Automated, System.
					
Initial State: Not applicable
					
Input: $A$ = $\begin{bmatrix} 
2 & 3 \\
4 & 9 
\end{bmatrix}$, $b$ = $\begin{bmatrix} 
6 \\
15 
\end{bmatrix}$

Output:4 $x$ = $\begin{bmatrix} 
3/2\\
1
\end{bmatrix}$, Success = true
					
How test will be performed: Automated system test
					
\item{\textbf{T-\refstepcounter{tnum}\thetnum \label{t-gaussian
elimination_three}: Linear system involving three equations and three
variables}}

Type: Functional, Automated, System.
					
Initial State: Not applicable
					
Input: $A$ =  $\begin{bmatrix} 
1 & 3 & -2 \\
3 & 5 & 6\\
2 & 4 & 3
\end{bmatrix}$, $b$ = $\begin{bmatrix} 
5\\
7\\
8 
\end{bmatrix}$
					
Output: $x$ = $\begin{bmatrix} 
-15\\
8\\
2 
\end{bmatrix}$, Success = true
					
How test will be performed: Automated system test

\item{\textbf{T-\refstepcounter{tnum}\thetnum \label{t-gaussian
elimination_five}: Linear system involving six equations and six variables}}

Type: Functional, Automated, System.
					
Initial State: Not applicable
					
Input: $A$ = $\begin{bmatrix} 
1 & 1 & -2 & 1 & 3 & -1 \\
2 & -1 & 1 & 2 & 2 & -3 \\
1 & 3 & -3 & -1 & 2 &1 \\
5 & 2 &-1 & -1 & 2 & 1 \\
-3 & -1 & 2 & 3 & 1 &3 \\
4 & 3 & 1 & -6 & -3 & -2
\end{bmatrix}$, $b$  = $\begin{bmatrix} 
 4\\
20\\
-15\\
-3\\
16\\
-27 
\end{bmatrix}$
					
Output: $x$ = $\begin{bmatrix} 
1/3\\
-430/99\\
313/99\\
104/99\\
142/33\\
-37/99 
\end{bmatrix}$, Success = true
					
How test will be performed: Automated system test

\item{\textbf{T-\refstepcounter{tnum}\thetnum \label{t-gaussian
elimination_three}: Linear system of equations which are singular}}

Type: Functional, Automated, System.
					
Initial State: Not applicable
					
Input: $A$ =$\begin{bmatrix} 
0 & 2 & -1 \\
3 & -2 & 1\\
3 & 2 & -1
\end{bmatrix}$, $b$ = $\begin{bmatrix} 
5\\
7\\
8 
\end{bmatrix}$
					
Output: $x$ = no solution, Success = true
					
How test will be performed: Automated system test

\end{enumerate}



\paragraph{Gauss-Jordan Method }

\begin{enumerate}

\item{\textbf{T-\refstepcounter{tnum}\thetnum \label{t-gauss-jordan_simple}:
Simple Linear system involving two equations and two variables}}


Type: Functional, Automated, System.
					
Initial State: Not applicable
					
Input: $A$ = $\begin{bmatrix} 
2 & 3 \\
4 & 9 
\end{bmatrix}$, $b$ = $\begin{bmatrix} 
6 \\
15 
\end{bmatrix}$

Output: $x$ = $\begin{bmatrix} 
3/2\\
1
\end{bmatrix}$, Success = true
					
How test will be performed: Automated system test
					
\item{\textbf{T-\refstepcounter{tnum}\thetnum \label{t-gauss-jordan
elimination_three}: Linear system involving three equations and three
variables}}

Type: Functional, Automated, System.
					
Initial State: Not applicable
					
Input: $A$ =  $\begin{bmatrix} 
1 & 3 & -2 \\
3 & 5 & 6\\
2 & 4 & 3
\end{bmatrix}$, $b$ = $\begin{bmatrix} 
5\\
7\\
8 
\end{bmatrix}$
					
Output: $x$ = $\begin{bmatrix} 
-15\\
8\\
2 
\end{bmatrix}$, Success = true
					
How test will be performed: Automated system test

\item{\textbf{T-\refstepcounter{tnum}\thetnum \label{t-gauss-jordan_six}: Linear
system involving six equations and six variables}}

Type: Functional, Automated, System.
					
Initial State: Not applicable
					
Input: $A$ = $\begin{bmatrix} 
1 & 1 & -2 & 1 & 3 & -1 \\
2 & -1 & 1 & 2 & 2 & -3 \\
1 & 3 & -3 & -1 & 2 &1 \\
5 & 2 &-1 & -1 & 2 & 1 \\
-3 & -1 & 2 & 3 & 1 &3 \\
4 & 3 & 1 & -6 & -3 & -2
\end{bmatrix}$, $b$  = $\begin{bmatrix} 
 4\\
20\\
-15\\
-3\\
16\\
-27 
\end{bmatrix}$
					
Output: $x$ = $\begin{bmatrix} 
1/3\\
-430/99\\
313/99\\
104/99\\
142/33\\
-37/99 
\end{bmatrix}$, Success = true
					
How test will be performed: Automated system test

\item{\textbf{T-\refstepcounter{tnum}\thetnum \label{t-gauss-jordan_three}: Linear system of equations which are singular}}

Type: Functional, Automated, System.
					
Initial State: Not applicable
					
Input: $A$ =$\begin{bmatrix} 
0 & 2 & -1 \\
3 & -2 & 1\\
3 & 2 & -1
\end{bmatrix}$, $b$ = $\begin{bmatrix} 
5\\
7\\
8 
\end{bmatrix}$
					
Output: $x$ = no solution, Success = true
					
How test will be performed: Automated system test

\end{enumerate}


\subsection{Tests for Nonfunctional Requirements}

\subsubsection{Performance Requirements}
		

\paragraph{Accuracy}

\begin{enumerate}

\item{\textbf{T-\refstepcounter{tnum}\thetnum \label{t-accuracy}: Calculating the
accuracy of Library of Linear Algebraic Equation Solver.}}


Type: Non-Functional, Automatic, Accuracy
					
Initial State: Not Applicable
					
Input/Condition: $A$ = $\begin{bmatrix} 
1 & 3 & -2 \\
3 & 5 & 6\\
2 & 4 & 3
\end{bmatrix}$, $b$ = $\begin{bmatrix} 
5\\
7\\
8 
\end{bmatrix}$
					
Output/Result: $\epsilon_{Relative}$ will be calculated by comparing the result
obtained by Library of Linear Algebraic Equation Solver and the RStudio's optR library
functional programs by the following equation as norm

\begin{center}
 $\epsilon_\text{rel} = \text{norm} = \frac{||x_\text{R} - x_\text{LAES}||}{||x_\text{R}||}$
\end{center}
					
How test will be performed: Automatic System Test.
					
\end{enumerate}



\subsection{Traceability Between Test Cases and Requirements}

The following table shows the traceability mapping for test case and the
Instance Models described in Commonality Analysis since CA does not include
requirements.

\begin{table} [H]
  \caption{Requirements Traceability Matrix}
  \label{Table:Table_Traceability}  
\begin{tabular}{|c|p{8cm}|}
  \hline	
  \textbf{Test Number} & \textbf{CA Requirements}\\
  \hline 
   T1& IM1\\ \hline
   T2& IM1\\ \hline
   T3& IM1\\ \hline
   T4& IM1\\ \hline
   T5& IM2\\ \hline
   T6& IM2\\ \hline
   T7& IM2\\ \hline
   T8& IM2\\ \hline
   
  

\end{tabular}\\
\end{table}

\section{Unit Testing plan}


\begin{enumerate}

\item{\textbf{T-\refstepcounter{tnum}\thetnum \label{t-accuracy}: Unit Testing.}}



The testing plan for Library of Linear Algebraic Equation Solver is divided into
several unique steps. These sequence of steps also includes specific function
unit tests. The code will be divided into several individual units of code. A
unit is a smallest component of code which can be tested. These units of code
are tested if they are fit to use or not.

The code will be designed in several modules like Input/Output module, Gaussian
Elimination Module, Gauss-Jordan Elimination Module. The basic plan for unit
testing is that I will be using RStudio's unit testing packages. The tests from
the functional requirements will also be used. The unit testing is a stage where
we should test for code coverage. My aim is for 100\% code coverage.
					

					
\end{enumerate}



\bibliographystyle{plainnat}

\bibliography{SRS}

\newpage

\section{Appendix}



\subsection{Symbolic Parameters}

The definition of the test cases will call for SYMBOLIC\_CONSTANTS.
Their values are defined in this section for easy maintenance.

\renewcommand{\arraystretch}{1.2}
\noindent \begin{longtable*}{l l p{12cm}} \toprule
\textbf{symbol} & \textbf{unit} & \textbf{description}\\
\midrule

$\epsilon$ & none & The measure of the difference between results obtained with
Library of Linear Algebraic Equation Solver and RStudio.\\

\bottomrule
\end{longtable*}




\end{document}
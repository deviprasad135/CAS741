\documentclass[12pt, titlepage]{article}

\usepackage{amsmath, mathtools}

\usepackage[round]{natbib}
\usepackage{amsfonts}
\usepackage{amssymb}
\usepackage{graphicx}
\usepackage{colortbl}
\usepackage{xr}
\usepackage{hyperref}
\usepackage{longtable}
\usepackage{xfrac}
\usepackage{tabularx}
\usepackage{float}
\usepackage{siunitx}
\usepackage{booktabs}
\usepackage{multirow} 
\usepackage[section]{placeins}
\usepackage{caption}
\usepackage{fullpage}
\usepackage{listings}

\hypersetup{
bookmarks=true,     % show bookmarks bar?
colorlinks=true,       % false: boxed links; true: colored links
linkcolor=red,          % color of internal links (change box color with linkbordercolor)
citecolor=blue,      % color of links to bibliography
filecolor=magenta,  % color of file links
urlcolor=cyan          % color of external links
}

\usepackage{array}

\input{../../Comments}

\newcommand{\progname}{Library of Linear Algebraic Equation Solver}

\begin{document}

\title{Module Interface Specification for Library of Linear Algebraic Equation Solver}

\author{Devi Prasad Reddy Guttapati}

\date{\today}

\maketitle

\pagenumbering{roman}

\section{Revision History}

\wss{You commented out the input for Comments.  I had to uncomment it so that I
  could add comments with the wss command.}

\begin{tabularx}{\textwidth}{p{3cm}p{2cm}X}
\toprule {\bf Date} & {\bf Version} & {\bf Notes}\\
\midrule
Date 1 & 1.0 & Initial Draft\\

\bottomrule
\end{tabularx}

~\newpage

\section{Symbols, Abbreviations and Acronyms}

See SRS Documentation at:  \url{https://github.com/deviprasad135/CAS741/blob/master/Doc/SRS/CA.pdf}


\newpage

\tableofcontents

\newpage

\pagenumbering{arabic}

\section{Introduction}

The following document details the Module Interface Specifications for Library
of Linear Algebraic Equation Solver.  \wss{You should usually avoid one sentence
  paragraphs.  These two paragraphs would certainly make sense combined
  together.}

Complementary documents include the System Requirement Specifications
and Module Guide.  The full documentation and implementation can be
found at the mentioned link below:

\url{https://github.com/deviprasad135/CAS741}.  

\section{Notation}

%\wss{You should describe your notation.  You can use what is below as
  %a starting point.}

The structure of the MIS for modules comes from \cite{hoffman1999software},
with the addition that template modules have been adapted from
\cite{ghezzi2002fundamentals}.  The mathematical notation comes from Chapter 3 of
\cite{hoffman1999software}.  For instance, the symbol := is used for a
multiple assignment statement and conditional rules follow the form $(c_1
\Rightarrow r_1 | c_2 \Rightarrow r_2 | ... | c_n \Rightarrow r_n )$.

The following table summarizes the primitive data types used by \progname. 

\begin{center}
\renewcommand{\arraystretch}{1.2}
\noindent 
\begin{tabular}{l l p{7.5cm}} 
\toprule 
\textbf{Data Type} & \textbf{Notation} & \textbf{Description}\\ 
\midrule
character & char & a single symbol or digit\\
integer & $\mathbb{Z}$ & a number without a fractional component in (-$\infty$, $\infty$) \\
natural number & $\mathbb{N}$ & a number without a fractional component in [1, $\infty$) \\
real & $\mathbb{R}$ & any number in (-$\infty$, $\infty$)\\
\bottomrule
\end{tabular} 
\end{center}

\noindent
The specification of \progname \ uses some derived data types: sequences, strings, and
tuples. Sequences are lists filled with elements of the same data type. Strings
are sequences of characters. Tuples contain a list of values, potentially of
different types. In addition, \progname \ uses functions, which
are defined by the data types of their inputs and outputs. Local functions are
described by giving their type signature followed by their specification.

\section{Module Decomposition}

The following table is taken directly from the Module Guide document for this project.

\begin{table}[h!]
\centering
\begin{tabular}{p{0.3\textwidth} p{0.6\textwidth}}
\toprule
\textbf{Level 1} & \textbf{Level 2}\\
\midrule

{Hardware-Hiding} & ~ \\
\midrule

\multirow{3}{0.3\textwidth}{Behaviour-Hiding} 
& Input Computing Module\\
& Output computing Module\\
& Library of Linear Algebraic Equation Solver Module\\


\midrule

\multirow{2}{0.3\textwidth}{Software Decision}  
& Matrix Module\\

& Gaussian Elimination Module\\

& Gauss-Jordan Elimination Module\\


\bottomrule

\end{tabular}
\caption{Module Hierarchy}
\label{TblMH}
\end{table}

\wss{Rather than Input Computing, you should just say Input.  You don't really
  compute the input.  Output is also a simpler name than Output Computing.}

\newpage
~\newpage

\section{MIS of {Library of Linear Algebraic Equation Solver Module}} \label{modllaes} %\wss{Use labels for cross-referencing}

\subsection{Module}

LLAES

%\wss{Short name for the module}

\subsection{Uses}
IC (Section \ref{modic}), OC (Section \ref{modoc}), Matrix (Section \ref{modmatrix}), GE (Section \ref{modge}), GJE (Section \ref{modgje})


\subsection{Syntax}

\begin{center}
\begin{tabular}{p{4cm} p{4cm} p{2cm} p{2cm}}
\hline
\textbf{Name} & \textbf{In} & \textbf{Out} & \textbf{Exceptions} \\
\hline
Linear\_Algebraic Equation\_Solving Methods & Linear\_Algebraic
Equation\_Solving Method $\in {1, 2}$& - & - \\

\hline
\end{tabular}
\end{center}

\wss{Rather than use 1, 2, you should introduce an enumerated type that has two
  values: GaussElim, GaussJordan.  You could export this new type, rather than
  define it in the input field}

\subsection{Semantics}

\subsubsection{State Variables}

None


\subsubsection{Access Routine Semantics}

\noindent %\wss{accessProg}():
\begin{itemize}
\item transition: None %\wss{if appropriate} 
\item output: None%\wss{if appropriate} 
\item exception: None%\wss{if appropriate} 
\item pseudocode: 
\begin{lstlisting}
if (option = 1)
   then solve using Gaussian Elimination Method
if (option = 2)
   then solve using Gauss-Jordan Elimination Method 
\end{lstlisting}
\end{itemize}

\wss{You have no transition, no output, no exceptions, so your module doesn't
  actually do anything.  What you enter under pseudocode should actually be
  entered under transition.}

\wss{You actually aren't passing any arguments to your solver methods.}

\newpage


\section{MIS of {Input Computing Module}} \label{modic} %\wss{Use labels for cross-referencing}

\subsection{Module}

IC

%\wss{Short name for the module}

\subsection{Uses}
Matrix (Section \ref{modmatrix}), GE (Section \ref{modge}), GJE (Section \ref{modgje}) 


\subsection{Syntax}

\begin{center}
\begin{tabular}{p{2cm} p{4cm} p{4cm} p{2cm}}
\hline
\textbf{Name} & \textbf{In} & \textbf{Out} & \textbf{Exceptions} \\
\hline
A & $n \times n$ matrix $\in$ $\mathbb{R}$ and n $> 0$ & - & Complex Numbers \\
b & $n \times 1$ matrix $\in$ $\mathbb{R}$ and n $> 0$ & - & Complex Numbers \\
\hline
\end{tabular}
\end{center}

\wss{Your type for A and b is not correct.  The type for a matrix can be
  represented as $\mathbb{R}^{n \times n}$}

\subsection{Semantics}

\subsubsection{State Variables}

None


\subsubsection{Access Routine Semantics}

\noindent %\wss{accessProg}():
\begin{itemize}
\item transition: None %\wss{if appropriate} 
\item output: None %\wss{if appropriate} 
\item exception: None %\wss{if appropriate} 
\end{itemize}

\wss{This module doesn't actually do anything!  There is not transition, not
  output and no exceptions.  Rather than an Input module, you can just use
  multiple instances of your Matrix module to define the input to your solver
  module.}

\newpage


\section{MIS of {Output Computing Module}} \label{modoc} %\wss{Use labels for cross-referencing}

\subsection{Module}

OC

%\wss{Short name for the module}

\subsection{Uses}

None

\subsection{Syntax}

\begin{center}
\begin{tabular}{p{2cm} p{4cm} p{4cm} p{2cm}}
\hline
\textbf{Name} & \textbf{In} & \textbf{Out} & \textbf{Exceptions} \\
\hline


x & - & $n \times 1$ matrix $\in$ $\mathbb{R}$ and n $> 0$ & - \\
\hline
\end{tabular}
\end{center}

\wss{The type for x is not expressed correctly.}

\subsection{Semantics}

\subsubsection{State Variables}

None

\subsubsection{Access Routine Semantics}

\noindent %\wss{accessProg}():
\begin{itemize}
\item transition: None%\wss{if appropriate} 
\item output: None%\wss{if appropriate} 
\item exception: None%\wss{if appropriate}
\item pseudocode: 

\begin{lstlisting}
if (displayresult)
   {print(x);}
end if
\end{lstlisting}

\end{itemize}

\wss{If you are printing output to the screen, or to a file, you need an
  environment variable.}

\newpage


\section{MIS of {Matrix Module}} \label{modmatrix} %\wss{Use labels for cross-referencing}

\subsection{Module}

Matrix

%\wss{Short name for the module}

\subsection{Uses}
GE (Section \ref{modge}), GJE (Section \ref{modgje}), OC (Section \ref{modoc})


\subsection{Syntax}

\begin{center}
\begin{tabular}{p{2cm} p{4cm} p{4cm} p{2cm}}
\hline
\textbf{Name} & \textbf{In} & \textbf{Out} & \textbf{Exceptions} \\
\hline
A & $\mathbb{R}$ & $n \times n$ matrix $\in$ $\mathbb{R}$ and n $> 0$ & - \\
b & $\mathbb{R}$ & $n \times 1$ matrix $\in$ $\mathbb{R}$ and n $> 0$ & - \\
\hline
\end{tabular}
\end{center}

\wss{You need to think about this module more.  The input cannot be a single
  real number.  The input is a sequence of real numbers.  You also need state
  variables.  You can look at the sequence module example in the lecture slides,
  combined with the examples for Abstract Data Types (ADTs).  For an Abstract
  Data Type you need to use the ``Template Module,'' rather than ``Module.''
  You don't actually represent A and b together in this module.  The Matrix
  module defines a new ADT.  You can create an A object and a b object.  This
  about the constructor for each matrix and about the getters that makes sense.}

\subsection{Semantics}

\subsubsection{State Variables}

None

\subsubsection{Access Routine Semantics}

\noindent %\wss{accessProg}():
\begin{itemize}
\item transition: None%\wss{if appropriate} 
\item output: None%\wss{if appropriate} 
\item exception: None%\wss{if appropriate} 
\item pseudocode:

\begin{lstlisting}
function{
         A = (Input of Real Numbers)
         n = $(length (A))^{1/2}$
         A = matrix(A, row = n, col = n, byrow = TRUE)
         }
\end{lstlisting}

\end{itemize}

\newpage


\section{MIS of {Gaussian Elimination  Module}} \label{modge}

\subsection{Module}

GE

%\wss{Short name for the module}

\subsection{Uses}
This module is used to solve the system of Linear Algebraic Equations.


\subsection{Syntax}

\begin{center}
\begin{tabular}{p{2cm} p{4cm} p{4cm} p{2cm}}
\hline
\textbf{Name} & \textbf{In} & \textbf{Out} & \textbf{Exceptions} \\
\hline

A & $n \times n$ matrix $\in$ $\mathbb{R}$ and n $> 0$ & - & always\_a\_square\_matrix no\_singular\_matrix\\
b & $n \times 1$ matrix $\in$ $\mathbb{R}$ and n $> 0$ & - & - \\
x & - & $n \times 1$ matrix $\in$ $\mathbb{R}$ and n $> 0$ & - \\
\hline
\end{tabular}
\end{center}

\wss{The type for A and b should be Matrix.  That is why you are defining a
  Matrix type.}

\subsection{Semantics}

\subsubsection{State Variables}

None

\subsubsection{Access Routine Semantics}

\noindent %\wss{accessProg}():
\begin{itemize}
\item transition: None%\wss{if appropriate} 
\item output: None%\wss{if appropriate} 
\item exception: None%\wss{if appropriate}
\item pseudocode:
\begin{lstlisting}[mathescape=true]
for $k = 1$ to $n-1$
    find a pivot p such that
    $|a_{pk}| \geq |a_{ik}|$ for $K \leq i \leq n$
    if $|a_{pk}| = 0$ do
        return "Singular Matrix"
        end the entire loop
    else 
        interchange row p and k
        
    for $i = k+1$ to $n$
        $factor_{ik} = \dfrac{a_{ik}}{a_{kk}}$
        for $j = k+1$ to $n$
            $a_{ij} = a_{ij} - factor_{ik} * a_{kj}$
        end for
    end for
end for

$x_n = \dfrac{b_n^{'}}{a_{nn}} $
for i in n-1 to 1
    for j in i+1 to n
        $sum = a_{ij}x_j$
    end for
    $x_i = \dfrac{b_n^{'} - sum} {a_{ii}}$
end for
\end{lstlisting}
\end{itemize}

\section{MIS of {Gauss-Jordan Elimination  Module}} \label{modgje}

\subsection{Module}

GJE

%\wss{Short name for the module}

\subsection{Uses}
This module is used to solve the system of Linear Algebraic Equations.


\subsection{Syntax}

\begin{center}
\begin{tabular}{p{2cm} p{4cm} p{4cm} p{2cm}}
\hline
\textbf{Name} & \textbf{In} & \textbf{Out} & \textbf{Exceptions} \\
\hline

A & $n \times n$ matrix $\in$ $\mathbb{R}$ and n $> 0$ & - &  always\_a\_square\_matrix no\_singular\_matrix \\
b & $n \times 1$ matrix $\in$ $\mathbb{R}$ and n $> 0$ & - & - \\
x & - & $n \times 1$ matrix $\in$ $\mathbb{R}$ and n $> 0$ & - \\
\hline
\end{tabular}
\end{center}

\subsection{Semantics}

\subsubsection{State Variables}

None

\subsubsection{Access Routine Semantics}

\noindent %\wss{accessProg}():
\begin{itemize}
\item transition: None%\wss{if appropriate} 
\item output: None%\wss{if appropriate} 
\item exception: None%\wss{if appropriate} 
\item pseudocode:
\begin{lstlisting}[mathescape=true]

for $k = 1$ to $n-1$
    find a pivot p such that
    $|a_{pk}| \geq |a_{ik}|$ for $K \leq i \leq n$
    if $|a_{pk}| = 0$ do
        return "Singular Matrix"
        end the entire loop
    else 
        interchange row p and k
        
    for $i = k+1$ to $n$
        $factor_{ik} = \dfrac{a_{ik}}{a_{kk}}$
        for $j = k+1$ to $n$
            $a_{ij} = a_{ij} - factor_{ik} * a_{kj}$
        end for
    end for
end for

Assuming that the matrix is not singular
for k = n to 2
    for i = k+1 to 1
        $factor_{ik} = \dfrac{a_{ik}}{a_{kk}}$
        for j = k-1 to 1
            $a_{ij} = a_{ij} - factor_{ik} * a_{kj}$
        end for
    end for
end for

for i in 1 to n
    $x_i = \dfrac{b_n^{'}} {a_{ii}}$
end for

\end{lstlisting}
\end{itemize}


\newpage

\bibliographystyle{plainnat}
\bibliography{ref}

\wss{Why did you redo the Bib entry for HoffmanAndStrooper1995?  You could have
  saved time (and improved the reference information) by just using the entry in
  the course repo, under References.bib.}

\newpage

\section{Appendix} \label{Appendix}
Not Applicable

%\wss{Extra information if required}

\end{document}
